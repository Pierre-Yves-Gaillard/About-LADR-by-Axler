% ladr-a-b
% https://docs.google.com/document/d/1KR9DXR7CVsgUr_-3u9efOHLQUj-rGq1IUS0jUWWuRnE/edit?tab=t.0
% !TEX encoding = UTF-8 Unicode
% https://www.google.com/search?q=current+date+and+time 7:15 a.m. Saturday, November 2, 2024 (EDT) Time in Montreal, QC
% https://www.overleaf.com/read/fbvxfjsvrgcv#afc0a5 https://github.com/Pierre-Yves-Gaillard/About-LADR-by-Axler/tree/main https://latexonline.cc/
\documentclass[11pt,letterpaper]{article}
\usepackage{fancyhdr}
\fancyhf{}
\renewcommand{\headrulewidth}{0pt}
\fancyfoot[C]{\thepage}
\fancyfoot[R]{{\tiny ladr-a-b,\ \filemodprintdate{\jobname},\ \filemodprinttime{\jobname},\ 1730546178}} % https://www.site24x7.com/tools/time-stamp-converter.html (https://www.unixtimestamp.com/ bad)
% pdflatex makeindex pdflatex pdflatex
% pdflatex bibtex pdflatex pdflatex
% name of the bib file: refs.bib
\usepackage[T1]{fontenc}
\usepackage[utf8]{inputenc}
\usepackage{amssymb,amsmath,amsthm} 
%\usepackage[letterpaper,top=30pt,left=40pt,right=40pt,bottom=60pt]{geometry}
\usepackage[letterpaper,top=50pt,left=60pt,right=60pt,bottom=80pt]{geometry}
%\usepackage{imakeidx}
%\usepackage{datetime}
\usepackage{filemod}%https://stackoverflow.com/questions/2118972/latex-command-for-last-modified
\usepackage{microtype}
%\usepackage{graphicx}%https://www.johndcook.com/blog/2020/11/18/rotating-symbols-in-latex/ see also patch240105g
%\usepackage{tikz-cd}
%\usepackage{varwidth}
%\usepackage{comment}
\usepackage[pdfusetitle]{hyperref}%\texorpdfstring{$\boldsymbol{\tau_2}$}{tau2}
%\usepackage{marvosym}% emoji https://tex.stackexchange.com/questions/3695/smileys-in-latex https://ctan.org/pkg/marvosym?lang=en
%\usepackage{showkeys}%\usepackage{showlabels}
\pagestyle{fancy}
%\pagestyle{empty}
\setlength{\parskip}{5pt} % variable
\renewcommand{\baselinestretch}{1.1} % variable
%\newtheorem{thm}{Theorem}%[section]
%\newtheorem{cor}[thm]{Corollary}
%\newtheorem{lem}[thm]{Lemma}
%\newtheorem{prop}[thm]{Proposition}
%\theoremstyle{definition}
%\newtheorem{conv}[thm]{Convention}
%\newtheorem{set}[thm]{Setting}
%\newtheorem{dfn}[thm]{Definition}
%\newcommand{\B}{\mathcal B}
%\newcommand{\bc}{\bigcup}
%\newcommand{\eps}{\varepsilon}%
%\newcommand{\id}{\mathrm{id}}
%\newcommand{\mc}{\mathcal}
\newcommand{\noi}{\noindent}%
%\newcommand{\Q}{\mathbb Q}%
\newcommand{\sm}{\setminus}
\newcommand{\R}{\mathbb R}
%\newcommand{\T}{\mathcal T}
\newcommand{\Z}{\mathbb Z}
\begin{document}% \tiny \scriptsize \footnotesize \small \normalsize \large \Large \LARGE \huge \Huge 
\begin{center}
{\LARGE A few comments about ``Linear Algebra Done Right'' by Axler}\bigskip 

Pierre-Yves Gaillard\footnote{ORCID https://orcid.org/0000-0002-7960-1698} 
\end{center}%\centerline{LADR}\bigskip 

\noi As the title indicates, we make a comments about the book \textbf{Linear Algebra Done Right} by Sheldon Axler. This is a work in progress. This is a complement to Jon U jubnoske08's excellent solutions in 

[1] \url{https://github.com/jubnoske08/linear_algebra/tree/main}%\bigskip 

\noi$\bullet$ \textbf{Exercise 1C9.} A function $f:\R\to\R$ is called \textbf{periodic} if there exists a positive number $p$ such that $f(x) = f(x+p)$ for all $x\in\R$.  Is the set of periodic functions from $\R$ to $\R$ a subspace of $\R^\R$?  Explain. 

\noi\textbf{Solution.} The answer is no. To prove this, consider the functions $f,g:\R\to\R$ defined by 
$$
f(x)=
\begin{cases}
1\text{ if }x\in\Z\\ 
0\text{ otherwise}
\end{cases}
$$ 
and 
$$
g(x)=
\begin{cases}
1\text{ if }x\in\sqrt2\ \Z\\ 
0\text{ otherwise}.
\end{cases}
$$ 
These functions are periodic (of period 1 and $\sqrt2$ respectively). Claim: $f+g$ is not periodic. The claim will imply that the set of periodic functions from $\R$ to $\R$ is not a subspace of $\R^\R$. Proof of the claim: We have $(f+g)(0)=2$. If $f+g$ admitted the period $p>0$, we would have $(f+g)(p)=2$, and there would be nonzero integers $m$ and $n$ such that $m=p=n\sqrt2$, and thus $\sqrt2=m/n$, contradiction. 


\noi$\bullet$ \textbf{Exercises 1C12 and 1C13.} Exercises 1C12 and 1C13 are particular cases of the following statement. 

\noi\textbf{Statement.} Let $n$ be a positive integer, let $K$ be a field containing at least $n$ distinct elements, let $V$ be a $K$-vector space, and let $V_1,\ldots,V_n$ be $n$ \emph{proper} subspaces of $V$. Then $V\ne V_1\cup\cdots\cup V_n$. 

\noi\textbf{Proof.} Assume by contradiction that we have $V=V_1\cup\cdots\cup V_n$. Given $v,w\in V$ with $v\ne w$ say that the \textbf{line} through $v$ and $w$ is the subset 
$$
L(v,w)=v+K\,(w-v)=\{v+a\,(w-v)\ |\ a\in K\}
$$ 
of $V$. Note that $L(w,v)=L(v,w)$. The lines $L(v,w)$ are easily seen to have the following properties: 

\noi(a) $v,w\in L(v,w)$, 

\noi(b) $L(v,w)$ contains at least $n$ distinct points\footnote{In this proof, the vectors of $V$ are called ``points''.}, 

\noi(c) if a subspace $W$ of $V$ has at least two points in common with $L(v,w)$, then $L(v,w)\subset W$ and $w-v\in W$. 

\noi We prove the statement. We can assume that $n$ is the least positive integer for which we have $V=V_1\cup\cdots\cup V_n$ for some family $(V_i)$ of proper subspaces. Clearly the integer $n$ is $\ge2$. We claim $V_n\subset V_1\cup\cdots\cup V_{n-1}$. This will imply $V=V_1\cup\cdots\cup V_{n-1}$, contradiction. To prove the claim, let $v_n$ be in $V_n$ and $v$ in $V\sm V_n$, and set $L:=L(v,v+v_n)$. We have $L\cap V_n=\varnothing$ because $v+av_n\in V_n$ would imply $v\in V_n$, contradiction. As a result, $L$ is contained in $V_1\cup\cdots\cup V_{n-1}$. By (b) and the pigeonhole principle, $L$ intersects some $V_i$ with $i<n$ in at least two points, and (c) yields $v_n\in V_i$, proving the claim. 

\end{document}

\end{document}